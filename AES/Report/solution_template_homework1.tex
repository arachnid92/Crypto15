\documentclass[11pt,twoside]{article}
\usepackage[T1]{fontenc}
\usepackage[latin1]{inputenc}
\usepackage[english, activeacute]{babel}
\usepackage{amsmath}
\usepackage{amscd}
\usepackage{amssymb}
\usepackage{multirow}
\usepackage{tabularx}
\usepackage{url}
\usepackage{fancyhdr}
\usepackage{lastpage}
\usepackage[a4paper,margin=2.5cm,hmarginratio=1:1]{geometry}

%%%%%%%%%%%%%%%%%%%%%%%%%%%%%%%%%%%%%%%%%%%%%%%%%%%%%%%%%%%%%%%%%%%%%%%%%%%
%%%%%%%%%%%%%% ENTER YOUR PERSONAL INFORMATION HERE %%%%%%%%%%%%%%%%%%%%%%%
%%%%%%%%%%%%%%%%%%%%%%%%%%%%%%%%%%%%%%%%%%%%%%%%%%%%%%%%%%%%%%%%%%%%%%%%%%%


% Your name, personal number, and email.
\newcommand{\firstname}{Manuel Osvaldo}
\newcommand{\lastnames}{Olguin Mu\~noz}
\newcommand{\persnr}{921223-4679}
\newcommand{\email}{molguin@dcc.uchile.cl}


% Your two study pals. Leave empty as necessary.
\newcommand{\studypalX}{}
\newcommand{\studypalXemail}{}
\newcommand{\studypalY}{}
\newcommand{\studypalYemail}{}


%%%%%%%%%%%%%%%%%%%%%%%%%%%%%%%%%%%%%%%%%%%%%%%%%%%%%%%%%%%%%%%%%%%%%%%%%%%
%%%%%%%%%%%%% DO NOT TOUCH ANYTHING BELOW THIS LINE %%%%%%%%%%%%%%%%%%%%%%%
%%%%%%%%%%%%%%%%%%%%%%%%%%%%%%%%%%%%%%%%%%%%%%%%%%%%%%%%%%%%%%%%%%%%%%%%%%%

\newcounter{problem}
\renewcommand\theproblem{\arabic{problem}}
\newenvironment{problem}{%
  \bigbreak
  \refstepcounter{problem}\noindent
  \llap{\textbf{\theproblem}\quad}\ignorespaces
}{%
  \par\if@nadaten@solutions\relax\else\filbreak\fi
}
\newenvironment{problem*}{%
  \bigbreak
  \refstepcounter{problem}\noindent
  \llap{\textbf{\theproblem}\quad}\ignorespaces
}{%
  \par
}
\newcounter{subproblem}[problem]
\renewcommand{\thesubproblem}{\arabic{problem}\alph{subproblem}}
\newenvironment{subproblem}{%
  \refstepcounter{subproblem}%
  \list{}{}%
  \item\leavevmode
  \llap{\hbox to \leftmargini{\textbf{\thesubproblem}\hfil}}%
  \ignorespaces
}{%
  \endlist\if@nadaten@solutions\relax\else\filbreak\fi
}
\newenvironment{subproblem*}{%
  \refstepcounter{subproblem}%
  \list{}{}%
  \item\leavevmode
  \llap{\hbox to \leftmargini{\textbf{\thesubproblem}\hfil}}%
  \ignorespaces
}{%
  \endlist
}

\newcommand{\homeworknr}{1}
\newcommand{\studentname}{\firstname~\lastnames}
\newcommand{\homework}{Homework~\homeworknr}
\newcommand{\coursenumber}{DD2448}
\newcommand{\coursename}{\coursenumber~Foundations of cryptography}
\newcommand{\coursenick}{krypto15}

\lhead[\studentname]{\coursename}
\chead{}
\rhead[\coursename]{\studentname}
\lfoot[\thepage~(\pageref{LastPage})]{}
\cfoot{}
\rfoot[]{\thepage~(\pageref{LastPage})}

\fancypagestyle{firststyle}
{
   \fancyhf{}
   \fancyfoot[R]{\thepage~(\pageref{LastPage})}
}

\renewcommand{\headrulewidth}{0pt}


%%%%%%%%%%%%%%%%%%%%%%%%%%%%%%%%%%%%%%%%%%%%%%%%%%%%%%%%%%%%%%%%%%%%%%%%%%%
%%% HERE YOU CAN ADD YOUR OWN MACROS AND ENVIRONMENTS IN THE PREAMBLE %%%%%
%%%%%%%%%%%%%%%%%%%%%%%%%%%%%%%%%%%%%%%%%%%%%%%%%%%%%%%%%%%%%%%%%%%%%%%%%%%

% Add your macros here.


\begin{document}

%%%%%%%%%%%%%%%%%%%%%%%%%%%%%%%%%%%%%%%%%%%%%%%%%%%%%%%%%%%%%%%%%%%%%%%%%%%
%%%%%%%%%%%% THE FOLLOWING GENERATES THE HEADER %%%%%%%%%%%%%%%%%%%%%%%%%%%
%%%%%%%%%%%% DO NOT TOUCH THIS %%%%%%%%%%%%%%%%%%%%%%%%%%%%%%%%%%%%%%%%%%%%
%%%%%%%%%%%%%%%%%%%%%%%%%%%%%%%%%%%%%%%%%%%%%%%%%%%%%%%%%%%%%%%%%%%%%%%%%%%

\thispagestyle{firststyle}

\noindent
\hspace{0.3cm}{\huge\textbf{\coursename}}

\noindent
\rule{\textwidth}{1pt}

\vspace{0.3cm}

\noindent
\begin{tabularx}{\textwidth}{lXl}
  \multirow{3}{*}{\textbf{\huge\homework}} && {\Large\textbf{\studentname}} \\
&&\\[-0.3cm]
  && {\Large\persnr} \\
&&\\[-0.35cm]
  && {\Large\texttt{\email}} \\
&&\\[-0.2cm]
\cline{3-3}
&&\\[-0.2cm]
  \multirow{2}{*}{\textbf{\huge\coursenick}} && {\small\studypalX, \texttt{\studypalXemail}} \\
&& {\small\studypalY, \texttt{\studypalYemail}}
\end{tabularx}

\vspace{0.2cm}
\noindent
\rule{\textwidth}{1pt}

\vspace{0.5cm}

\pagestyle{fancy}

%%%%%%%%%%%%%%%%%%%%%%%%%%%%%%%%%%%%%%%%%%%%%%%%%%%%%%%%%%%%%%%%%%%%%%%%%%%
%%%%%%%%%%%%%%%%%%%%% YOUR SOLUTIONS START HERE %%%%%%%%%%%%%%%%%%%%%%%%%%%
%%%%%%%%%%%%%%%%%%%%%%%%%%%%%%%%%%%%%%%%%%%%%%%%%%%%%%%%%%%%%%%%%%%%%%%%%%%
%%                                                                       %%
%%  Do NOT remove any problem-, or subproblem environments below. If     %%
%%  you can not solve a problem, then you MUST simply leave the "NOT     %%
%%  SOLVED" string intact. This ensures that the numbering is correct    %%
%%  and it simplifies grading, leaving more time to prepare lectures     %%
%%  and help students.                                                   %%
%%                                                                       %%
%%%%%%%%%%%%%%%%%%%%%%%%%%%%%%%%%%%%%%%%%%%%%%%%%%%%%%%%%%%%%%%%%%%%%%%%%%%

\begin{problem}
  \begin{subproblem}
    (5T) NOT SOLVED % We leave this place holder here for improved readability.
  \end{subproblem}
  \begin{subproblem}
    (1T) NOT SOLVED % We leave this place holder here for improved readability.
  \end{subproblem}
\end{problem}

\begin{problem}
  (2T) NOT SOLVED % We leave this place holder here for improved readability.
\end{problem}

\begin{problem}

  \textit{Handed in through Kattis.} ``Borrowed'' ideas:
  \begin{itemize}
    \item Implementation of the S-Box and R-Con as static arrays, since they are input independent and only take 256 different values.
    \item Multiplication in $GF(2^8)$ done through bit arithmetic:
    \begin{itemize}
      \item Multiplication by two: if the high bit of the input $X$ is $1$, shift one bit to the left and $XOR$ with $00011011$. Otherwise, only shift one bit left.
      \item Multiplication by three: multiply by two and $XOR$ with the original input.
    \end{itemize}
  \end{itemize}
  
\end{problem}

\begin{problem}
  \begin{subproblem}
    The information is sufficient to calculate the entropy of $Y$. Since its clear that $X_i, ..., X_n$ are mutually independent, one can conclude that $Y$ distributes uniformly over $(0, 1)^{n}$, and thus its entropy is easily calculated.
  \end{subproblem}
  \begin{subproblem}
    (1T) NOT SOLVED % We leave this place holder here for improved readability.
  \end{subproblem}
  \begin{subproblem}
    (1T) NOT SOLVED % We leave this place holder here for improved readability.
  \end{subproblem}
  \begin{subproblem}
    (2T) NOT SOLVED % We leave this place holder here for improved readability.
  \end{subproblem}
  \begin{subproblem}
    (2T) NOT SOLVED % We leave this place holder here for improved readability.
  \end{subproblem}
  \begin{subproblem}
    (2T) NOT SOLVED % We leave this place holder here for improved readability.
  \end{subproblem}
\end{problem}

\begin{problem}
  The Hill cipher is vulnerable to a known-plaintext attack, since the encryption is completly linear. For this attack, the adversary needs to have captured at least $m$ distinct plaintext-ciphertext pairs, where $m \times m$ is the size of the key used for the encryption (if $m$ is unknown to the attacker, he or she can continue trying with different values until the plaintext-ciphertext pairs agree - assuming $m$ is not too big). The time complexity of this algorithm is aproximately similar to that of matrix multiplication, which can be done using the Strassen algorithm in $O(n^{2.8074})$.

\end{problem}

\begin{problem}
  NOT COMPLETE \\
  
  First of all, given that these are all students with no previous knowledge of cryptography and none of them is a genious, I would assume all of the ciphers fall into one of the following categories:
  \begin{itemize}
    \item Substitution ciphers.
    \item Transposition ciphers.
    \item Very simple symmetric ciphers.
  \end{itemize}
  
  The program written to break these would then work in the following manner:
  \begin{enumerate}
    \item Feed two known, and different, plaintexts to the cipher, and wait for the outputs.
    \item Analyze the outputs, comparing them to the inputs:
    \begin{enumerate}
      \item If input and output have exactly the same set of characters (excluding spaces), but in different order, mark as transposition cipher.
      \item If input and output have different sets of characters, but the frequencies of characters are the same (for example, input has 3 A's, 2 B's, 5 E's and so on, and output has the same frequencies but with 3 W's, 2 R's, and so on), mark as substition cipher.
    \end{enumerate}
  \end{enumerate}
  
\end{problem}

\begin{problem}
  \begin{subproblem}
    Basically, an uniform adversary is a probabilistic polynomial-time Turing Machine, which operates in an ``uniform'' way over all values of the security parameter of a scheme. 
    On the other hand, a non-uniform adversary is a ``family of circuits''. For each value of the security parameter of a scheme, a different program in said family is executed - in a way, it adapts to the security parameter of the scheme.
  \end{subproblem}
  \begin{subproblem}
    Yes, it matters. Non-uniform adversaries are stronger than uniform ones, since the former can do everything the latter can, and more. Thus, security measures against non-uniform adversaries are always \textbf{stronger}, but often require stronger computational hardness assumptions.
  \end{subproblem}
\end{problem}
\newpage
\begin{problem}
  \begin{subproblem}
    Handed in on a separate sheet of paper.
  \end{subproblem}
  \begin{subproblem}
    For $t = 1$, we have
    \[ L_1 = R_0 \]
    \[ R_1 = L_0 \oplus f_1(R_0) \]
    Thus, the whole left half of the ciphertext ALWAYS corresponds to exactly the whole right half of the plaintext - an ocurrence which is extremely unlikely in any pseudorandom permutation.
  \end{subproblem}
  \begin{subproblem}
    For $t = 2$, we have
    \[ L_2 = R_1 = L_0 \oplus f_1(R_0) \]
    \[ R_2 = L_1 \oplus f_2(R_1) = R_{0} \oplus f_{2}(L_0 \oplus f_1(R_0)) \]
    
    If we now take two plaintexts with the same right half, $L^{a}_0R$ and $L^{b}_0R$, note that the left sides of their respective ciphertexts are 
    \[ L^{a}_2 = L^{a}_0 \oplus f_{1}(R) \]
    \[ L^{b}_2 = L^{b}_0 \oplus f_{1}(R) \]
    
    And thus the following is always true,
    \[ L^{a}_2 \oplus L^{b}_2 = L^{a}_0 \oplus L^{b}_0 \]
    
    which is very unlikely in a pseudo-random permutation.
  \end{subproblem}
  \begin{subproblem}
    
    For $t = 3$, we have
    \[ L_3 = R_2 = R_{0} \oplus f_{2}(L_0 \oplus f_1(R_0)) \]
    \[ R_3 = L_2 \oplus f_3(R_2) = (L_0 \oplus f_1(R_0)) \oplus f_3(R_{0} \oplus f_{2}(L_0 \oplus f_1(R_0)))  \]
    
    Now, if let's modify the left side of the message. We end up thus with $L_0 \oplus \xi || R_0$, and the ciphertext:
    
    \[ L_3' = R_{0} \oplus f_{2}(L_0 \oplus \xi \oplus f_1(R_0)) \]
    \[ R_3' = (L_0 \oplus \xi \oplus f_1(R_0)) \oplus f_3(R_{0} \oplus f_{2}(L_0 \oplus \xi \oplus f_1(R_0)))  \]
    
    Working in the other direction, we define $L_0''||R_0''$ such that its ciphertext has the same left side as $L_3'||R_3'$ and its right side is the same right side $XOR$ our modification $\xi$;
    
    \[ L_3'' = L_3' \]
    \[ R_3'' = R_3' \oplus \xi \]
    
    Then, we can derive as follows:
    
    \[ L_2'' = R_3'' \oplus f_3 ( L_3'' = L_3' ) = L_2 \]
    \[ L_1'' = R_2'' \oplus f_2 ( L_2'' = L_2 ) = L_3'' \oplus f_2 ( L_2 ) = R_0'' \]
    Also
    \[ R_0 = L_3 \oplus f_2 ( L_2 ) \]
    
    \[ \Rightarrow R_0'' \oplus L_3'' = R_0 \oplus L_3 = f_2(L_2) \]
    
    Which always hold! This couldn't be the case if this was a pseudorandom permutation.\\
    $\therefore$ A 3-round Feistel network is not a pseudorandom permutation.
    
    
  \end{subproblem}
\end{problem}

\end{document}
